% -*- compile-command: "pdflatex lab4.tex"-*-
\documentclass[a4paper,12pt]{article}
\usepackage[utf8]{inputenc}
\usepackage[swedish]{babel}
\usepackage{listings}
\begin{document}


\section{Lab 4}
\subsection{Load and inspect the heightmap}
\subsubsection{What kind of data does the heightmap image data hold? What range is the heightmap data?}
It contains integers between 0 and 255.

\subsubsection{The terrain is 4x4. How many polygons does it draw?}
It paints 18 triangles, one polygon.


\subsection{Navigating the heightmap}
\subsubsection{Did you need to extend your camera movement code? How?}
Yes, we made it actually work.

\subsection{Calculate normal vectors and apply lighting}
\subsubsection{How did you implement the cross product? Function call, inline...? Normalization? }
We calculated two vectors in each polygon and took the cross product between them and normalized the normal. We used the crossproduct provided in VectorUtils3.

\subsubsection{Which method did you use to find the normal vector?}
We used the above described method.

\subsection{Calculating map height for a point}

\subsubsection{How did you figure out what surface to use? }
We examined which corner in the quadrent the point was closest to by checking if the sum of the differenses between the point and the corner was larger then 1. 

\subsubsection{How did you calculate the height from the surface?}
Through the plane equation Ax + Bx + Cx = D where [A, B, C] is the normal to the surface. 

\subsection{Multitextured terrain}

\subsubsection{What kind of multitexturing function did you implement? }
We implemented different textures for different hights. 

\end{document}
