% -*- compile-command: "pdflatex lab3.tex"-*-
\documentclass[a4paper,12pt]{article}
\usepackage[utf8]{inputenc}
\usepackage[swedish]{babel}
\usepackage{listings}
\begin{document}


\section{Lab 3}
\subsection{Hierarcical modelling, the windmill}
\subsubsection{How can you get all four blades to rotate with just one time-dependent rotation matrix?}
Using the magic of M\_PI.

$
		mat4 rotBlade = Mult(Rz(M_PI / 2 * i + t), Ry(M_PI / 2));
$

\subsubsection{How do you make the wings follow the body's movements?}
We constructed the translation matrix of the wings by adding the translation matrix of the walls to a new matrix. Though MatrixAdd is working out very oddly for us.

\subsubsection{You may have to tweak some numbers to make the placement right. How can you do that without making a lot of small changes in numbers in the code followed by recompilations?}

In main we first run InitKeymapManager() and then in our update loop we have the following code:

\begin{lstlisting}
	if(keyIsDown('w'))
		y += 0.1;
	else if (keyIsDown('s'))
		y -= 0.1;
	if(keyIsDown('a'))
		x -= 0.1;
	else if(keyIsDown('d'))
		x += 0.1;
	if(keyIsDown('q'))
		z += 0.1;
	else if(keyIsDown('e'))
		z -= 0.1;

	if (keyIsDown('p'))
		printf("%f %f %f \n", x, y, z);

	transBlade = T(x, y, z);
\end{lstlisting}

\subsection{Manual viewing controls}
\subsubsection{What kind of control did you implement?}
We implemented a rotation around the object based on the X-axis of the
mouse. Moving the mouse in the Y-axis moves the camera along the
Y-axis of the world. In practice we implemented cylindrical
coordinates.

\subsubsection{Can you make this kind of control in some other way
  than manipulating a "look-at" matrix?}
Probably, reasonably, yes. We could move every object in the world instead.

\subsection{Virtual world and skybox}
\subsubsection{How did you handle the camera matrix for the skybox?}
We constructed it using the following code:
\begin{lstlisting}[float,label=lst:cameramatrix,caption=nextHopInfo: Camera matrix code]
  	drawObject(T(cameraPos.x, cameraPos.y, cameraPos.z), skybox);
\end{lstlisting}
This centers the skybox around the camera.

\subsubsection{How did you represent the objects? Is this a good way to manage a scene or would you do it differently for a ``real'' application?}
We have not grouped together a model with a texture and translation matrix, making our code not scale very well to bigger projects.

\subsubsection{What special considerations are needed when rendering a skybox?}
We turned off the Z-buffer.

\subsection{Specular shading, external light sources}
\subsubsection{How do you generate a vector from the surface to the eye? }
\begin{lstlisting}[float,label=lst:label,caption=Surface to eye]
vec3 eyeDirection =
  normalize(mat3(transform) * (-vert_surface));
\end{lstlisting}

\subsubsection{Which vectors need renormalization in the fragment shader? }
We normalize all directions which we modif. For example eyeDirection.


\subsection{Multitexturing}
\subsubsection{How did you choose to combine the texture colour and the lighting colour?}
Multiplication. We multiply the texture color by the lighting color.

\subsubsection{How did you choose to combine the two textures?}
We blended between two textures using the code in listing~\ref{lst:lol}.

\begin{lstlisting}[float,label=lst:lol,caption=Fragment shader]
out_Color = sin(vert_surface.x) *
  texture(maskrosen, vert_TexCoord) * vec4(light, 1.0) +
  (1 - sin(vert_surface.x)) *
  texture(texUnit, vert_TexCoord) * vec4(light, 1.0);
\end{lstlisting}

\end{document}