\documentclass[a4paper,12pt]{article}
\usepackage[utf8]{inputenc}
\usepackage[swedish]{babel}
\begin{document}
\section{Lab 1}

\subsection{Setup and getting acquainted with the lab shell}
\subsubsection{Where is the origin placed in the on-screen coordinate system?}
The coordinate system seems to be centered at the middle of the screen.

\subsubsection{Which direction are the X and Y axes pointing in the on-screen coordinate system?}
The X-axis increases upwards and the Y-axis increases to the right.

\subsubsection{The triangle color is controlled from the fragment shader. Would it be possible to control it from the main program? How?}
We could perhaps pass a color to the fragment shader, perhaps through the vertex shader. The color is deceded by a four dimensional vector. The fields are (R, G, B, A).

\subsection{Transformations in the vertex shader}
\subsubsection{What is the purpose of the ``in'', ``out'' and ``uniform'' modifiers?}
\begin{itemize}
\item[in] Defines input variables to the shader program.
\item[out] Defines output variablet of the shader program.
\item[uniform] A constant value writable by OpenGL but constant for fragment and vertex shaders.
\end{itemize}

\subsubsection{What is the output of the vertex shader?}
The output of our vertex shader is the position of the vertexes in the scene.

\subsubsection{What does the function glUniformMatrix4fv do?}
It defines a global Uniform accessable by the different shaders. It basically uploads them to the GPU.

\subsection{Simple animation}
\subsubsection{What is the frame rate of the animation?}
The frame rate is 50 frames per second.

\subsection{Color shading}
\subsubsection{Did you need to do anything different when uploading the color data?}
No, we do it in the exact same way.

\subsubsection{The "in" and "out" modifiers are now used for something different. What?}
They are exactly the same.

\subsubsection{What is this kind of shading called? What could we use otherwise?}
Color interpolation, textures could instead be used.


\subsection{Building a cube, visible surface detection}
\subsubsection{What problems did you encounter while building the cube?}
Render ordering issues, solved by the z-buffer.

\subsubsection{How do you change the facing of a polygon?}
We changed which side of the cube faced the viewport by using the z-buffer.

\subsection{Load a 3D model from disc}
\subsubsection{Why do we need normal vectors for a model?}
We need the normals to know which direction is the outward direction of the faces.

\subsubsection{What did you do in your fragment shader?}
  $out\_Color = vec4(in_colors, 0.5);$

\subsubsection{Should a normal vector always be perpendicular to a certain triangle? If not, why?}
It should be perpendicular to it's own triangle, otherwise it's not a normal.

\subsubsection{Now we are using glBindBuffer and glBufferData again. They deal with buffers, but in what way?}
\begin{itemize}
\item[glBindBuffer] Tells OpenGL which buffer we're currently operating on.
\item[glBufferData] Tells OpenGL how big a buffer we want created and then loads data into the buffer.
\end{itemize}


\end{document}